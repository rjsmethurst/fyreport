\documentclass{mn2e}
\usepackage{footnote}
\usepackage{graphicx}
\usepackage{amsmath}
\usepackage{natbib}
\usepackage{array}
\usepackage{color}
\usepackage{url}

\begin{document}
\title[The Star Formation History of the Green Valley]{Galaxy Zoo: Evidence for Diverse Star Formation Histories through the Green Valley}
\author[Smethurst et al. 2014]{R. ~J. ~Smethurst,$^1$
\\ $^1$ Oxford Astrophysics, Department of Physics, University of Oxford, Denys Wilkinson Building, Keble Road, Oxford, OX1 3RH, UK }

\maketitle

\begin{abstract}

\end{abstract}

\section{Introduction}
Previous large scale surveys of galaxies have revealed a bimodality in the colour-magnitude diagram (CMD) with two distinct populations; one at relatively low mass, with blue optical colours and another at relatively high mass, with red optical colours \citep{Baldry04, Baldry06, Willmer06, BLB08, Brammer09}. These populations were dubbed the `blue cloud' and `red sequence' respectively. The Galaxy Zoo project \citep{Lintott11}, which incorporated morphological classifications for a million galaxies revealed that this bimodality is not entirely morphology driven \citep{Bamford09, Skibba09}, detecting spiral galaxies in the red sequence \citep{Masters10} and elliptical galaxies in the blue cloud \citep{Sch09}.  

The sparsely populated colour space between these two populations, the so-called `green valley', provides clues to the nature and duration of galaxies' transitions from blue to red. This transition must therefore occur on rapid timescales, otherwise we would find an accumulation of galaxies residing in the green valley, rather than an accumulation in the red sequence as is observed \citep{Arnouts07, Martin07}. Green valley galaxies have therefore long been thought of as the `crossroads' of galaxy evolution; a transition population between the two main galactic stages of the star forming blue cloud and the `dead' red sequence \citep{Bell04, Wyder07, Schim07, Martin07, Faber07, Mendez11, Gonc12, Sch2014}. 

The intermediate colours of these green valley galaxies have been interpreted as evidence for recent quenching (suppression) of star formation \citep{Salim07}. Star forming galaxies are observed to lie on a well defined mass-SFR relation, however quenching a galaxy causes it to depart from this relation (\citealt{Noeske07, Peng}; see Figure \ref{sfr_mass_sub})

By studying the galaxies which  have just left this mass-SFR relation, we can probe the quenching mechanisms by which this occurs. There have been many previous theories for the initial triggers of these quenching mechanisms, including negative feedback from AGN \citep{Sch07}, mergers \citep{Darg10a}, supernovae winds \citep{MFB12} and secular evolution \citep{Masters10, Masters11}. By investigating the \emph{amount} of quenching that has occurred between the blue cloud, green valley and red sequence (the three populations) we can apply some constraints to these theories. 

We have been motivated by a recent result suggesting two contrasting evolutionary pathways through the green valley by different morphological types (Schawinski et al. 2014), specifically that late-type galaxies quench very slowly and form a nearly static disc population in the green valley, whereas early-type galaxies quench very rapidly, transitioning through the green valley and onto the red sequence in $\sim 1$~Gyr. That study used a toy model to examine quenching across the green valley; here we implement a novel method utilising Bayesian statistics (for a comprehensive overview of Bayesian statistics see either \citealt{MacKay} or \citealt{Sivia}) in order to find the most likely model description of the star formation histories of galaxies in the three populations. It also provides a direct comparison with our current understanding of galaxy evolution from stellar population synthesis (SPS, see section \ref{SPS}) models. 

%We have been motivated by a previous Galaxy Zoo investigation by  \cite[hereafter S14]{Sch2014}, who by using a toy model found two contrastingly different evolutionary pathways between morphological types across the green valley. Their conclusions suggested that late-type (disc-like) galaxies quench very slowly due to gas depletion across the blue cloud until they reach the green valley after several gigayears with little morphological changes (suggesting a static disc population in the green valley); whereas early-type (smooth-like) galaxies quench very rapidly, triggering a morphological change and transitioning to the red sequence in $\sim 1~\rm{Gyr}$. Unlike this previous study, this investigation implements a novel method utilising Bayesian statistics (for a comprehensive overview of Bayesian statistics see either \citealt{MacKay} or \citealt{Sivia}) in order to find the most likely model description of the star formation histories of galaxies in the three populations. It also provides a direct comparison with our current understanding of galaxy evolution from stellar population synthesis (SPS, see section \ref{SPS}) models. 

\begin{table*}
\begin{tabular*}{0.9\textwidth}{r| @{\extracolsep{\fill}}cccc}
\hline
\begin{tabular}[c]{@{}c@{}} {\color{white} -} \\ {\color{white} -}  \end{tabular} & All                                                      & Red Sequence                                              & Green Valley                                              & Blue Cloud \\  \hline 
Smooth-like ($p_s > 0.5$)        & \begin{tabular}[c]{@{}c@{}}42453\\ (33.6\%)\end{tabular} & \begin{tabular}[c]{@{}c@{}}17424\\ (13.8\%)\end{tabular}  & \begin{tabular}[c]{@{}c@{}}10687\\ (8.4\%)\end{tabular}   & \begin{tabular}[c]{@{}c@{}}14342\\ (11.3\%)\end{tabular}  \\ 
Disc-like ($p_d > 0.5$)          & \begin{tabular}[c]{@{}c@{}}83863\\ (66.4\%)\end{tabular} & \begin{tabular}[c]{@{}c@{}}10722\\ (8.4\%)\end{tabular}   & \begin{tabular}[c]{@{}c@{}}13257\\ (10.5\%)\end{tabular}  & \begin{tabular}[c]{@{}c@{}}59884\\ (47.4\%)\end{tabular}  \\
Early-type ($p_s \geq 0.8$) & \begin{tabular}[c]{@{}c@{}}10517\\ (8.3\%)\end{tabular}  & \begin{tabular}[c]{@{}c@{}}5337\\ (4.2\%)\end{tabular}    & \begin{tabular}[c]{@{}c@{}}2496\\ (2.0\%)\end{tabular}    & \begin{tabular}[c]{@{}c@{}}2684\\ (2.1\%)\end{tabular}    \\
Late-type ($p_s \geq 0.8$)  & \begin{tabular}[c]{@{}c@{}}51470\\ (40.9\%)\end{tabular} & \begin{tabular}[c]{@{}c@{}}4493\\ (3.6\%)\end{tabular}    & \begin{tabular}[c]{@{}c@{}}6817\\ (5.4\%)\end{tabular}    & \begin{tabular}[c]{@{}c@{}}40430\\ (32.0\%)\end{tabular}  \\ \hline
\textbf{Total}                       & \begin{tabular}[c]{@{}c@{}}\textbf{126316} \\ (100.0\%)\end{tabular}                                                & \begin{tabular}[c]{@{}c@{}}28146 \\ (22.3\%)\end{tabular} & \begin{tabular}[c]{@{}c@{}}23944 \\ (18.9\%)\end{tabular} & \begin{tabular}[c]{@{}c@{}}74226 \\ (58.7\%)\end{tabular} \\\hline
\end{tabular*}
\caption{Table showing the break down of the GZ2 sample into the subsets of the colour-magnitude diagram by galaxy type.}
\label{subs}
\end{table*}

Through this approach, we aim to determine the following:
\begin{enumerate}
\item What previous star formation history (SFH) causes a galaxy to reside in the green valley at the current epoch?
\item Why is the green valley so sparsely populated?
\item Is the green valley a transitional or static population? 
\item If the green valley is a transitional population then how many routes through it are there? 
\item Are there morphological dependant differences between these routes through the green valley? 
\end{enumerate}

This report proceeds as follows. Section \ref{data} contains a description of the sample data, which is used in the Bayesian analysis of an exponentially declining star formation history model, all described in Section \ref{models}. Section \ref{results} contains the results produced by this analysis, with Section \ref{diss} providing a detailed discussion of the results obtained. We also conclude our findings in Section \ref{conc}. The zero points of all ugriz magnitudes are in the AB system and where necessary we adopt the WMAP Seven-Year Cosmological parameters \citep{WMAP} with $(\Omega_m, \Omega_{\lambda}, h) = (0.26, 0.73, 0.71)$. 

\end{document}